
%% bare_jrnl.tex
%% V1.4b
%% 2015/08/26
%% by Michael Shell
%% see http://www.michaelshell.org/
%% for current contact information.
%%
%% This is a skeleton file demonstrating the use of IEEEtran.cls
%% (requires IEEEtran.cls version 1.8b or later) with an IEEE
%% journal paper.
%%
%% Support sites:
%% http://www.michaelshell.org/tex/ieeetran/
%% http://www.ctan.org/pkg/ieeetran
%% and
%% http://www.ieee.org/

%%*************************************************************************
%% Legal Notice:
%% This code is offered as-is without any warranty either expressed or
%% implied; without even the implied warranty of MERCHANTABILITY or
%% FITNESS FOR A PARTICULAR PURPOSE! 
%% User assumes all risk.
%% In no event shall the IEEE or any contributor to this code be liable for
%% any damages or losses, including, but not limited to, incidental,
%% consequential, or any other damages, resulting from the use or misuse
%% of any information contained here.
%%
%% All comments are the opinions of their respective authors and are not
%% necessarily endorsed by the IEEE.
%%
%% This work is distributed under the LaTeX Project Public License (LPPL)
%% ( http://www.latex-project.org/ ) version 1.3, and may be freely used,
%% distributed and modified. A copy of the LPPL, version 1.3, is included
%% in the base LaTeX documentation of all distributions of LaTeX released
%% 2003/12/01 or later.
%% Retain all contribution notices and credits.
%% ** Modified files should be clearly indicated as such, including  **
%% ** renaming them and changing author support contact information. **
%%*************************************************************************


% *** Authors should verify (and, if needed, correct) their LaTeX system  ***
% *** with the testflow diagnostic prior to trusting their LaTeX platform ***
% *** with production work. The IEEE's font choices and paper sizes can   ***
% *** trigger bugs that do not appear when using other class files.       ***                          ***
% The testflow support page is at:
% http://www.michaelshell.org/tex/testflow/



\documentclass[journal]{IEEEtran}
%
% If IEEEtran.cls has not been installed into the LaTeX system files,
% manually specify the path to it like:
% \documentclass[journal]{../sty/IEEEtran}

\usepackage{xeCJK} % CJK语言环境,使用XeLaTex进行编译
\usepackage{authblk} % 对应中文部分的作者机构特殊语法

% 将 authblk 包中的作者连词 and 替换为中文逗号
\renewcommand*{\Authsep}{,}
\renewcommand*{\Authand}{,}
\renewcommand*{\Authands}{,}

% *** GRAPHICS RELATED PACKAGES ***
%
\ifCLASSINFOpdf
  % \usepackage[pdftex]{graphicx}
  % declare the path(s) where your graphic files are
  % \graphicspath{{../pdf/}{../jpeg/}}
  % and their extensions so you won't have to specify these with
  % every instance of \includegraphics
  % \DeclareGraphicsExtensions{.pdf,.jpeg,.png}
\else
  % or other class option (dvipsone, dvipdf, if not using dvips). graphicx
  % will default to the driver specified in the system graphics.cfg if no
  % driver is specified.
  % \usepackage[dvips]{graphicx}
  % declare the path(s) where your graphic files are
  % \graphicspath{{../eps/}}
  % and their extensions so you won't have to specify these with
  % every instance of \includegraphics
  % \DeclareGraphicsExtensions{.eps}
\fi
% graphicx was written by David Carlisle and Sebastian Rahtz. It is
% required if you want graphics, photos, etc. graphicx.sty is already
% installed on most LaTeX systems. The latest version and documentation
% can be obtained at: 
% http://www.ctan.org/pkg/graphicx
% Another good source of documentation is "Using Imported Graphics in
% LaTeX2e" by Keith Reckdahl which can be found at:
% http://www.ctan.org/pkg/epslatex
%
% latex, and pdflatex in dvi mode, support graphics in encapsulated
% postscript (.eps) format. pdflatex in pdf mode supports graphics
% in .pdf, .jpeg, .png and .mps (metapost) formats. Users should ensure
% that all non-photo figures use a vector format (.eps, .pdf, .mps) and
% not a bitmapped formats (.jpeg, .png). The IEEE frowns on bitmapped formats
% which can result in "jaggedy"/blurry rendering of lines and letters as
% well as large increases in file sizes.
%
% You can find documentation about the pdfTeX application at:
% http://www.tug.org/applications/pdftex




% correct bad hyphenation here
\hyphenation{op-tical net-works semi-conduc-tor}


\begin{document}
%
% paper title
% Titles are generally capitalized except for words such as a, an, and, as,
% at, but, by, for, in, nor, of, on, or, the, to and up, which are usually
% not capitalized unless they are the first or last word of the title.
% Linebreaks \\ can be used within to get better formatting as desired.
% Do not put math or special symbols in the title.
\title{基于YOLOv8的头盔检测模型\\设计与实现}
%
%
% author names and IEEE memberships
% note positions of commas and nonbreaking spaces ( ~ ) LaTeX will not break
% a structure at a ~ so this keeps an author's name from being broken across
% two lines.
% use \thanks{} to gain access to the first footnote area
% a separate \thanks must be used for each paragraph as LaTeX2e's \thanks
% was not built to handle multiple paragraphs
%

\author{陈邵杰,郭昊,蔡明珠,刘政,黄俊毅,王晶昊,杨双菁,田睿朴,苗馨月% <-this % stops a space
\thanks{感谢电子科技大学机器学习课程闫老师与张老师的教学。}% <-this % stops a space
\thanks{2024年10月}}

% note the % following the last \IEEEmembership and also \thanks - 
% these prevent an unwanted space from occurring between the last author name
% and the end of the author line. i.e., if you had this:
% 
% \author{....lastname \thanks{...} \thanks{...} }
%                     ^------------^------------^----Do not want these spaces!
%
% a space would be appended to the last name and could cause every name on that
% line to be shifted left slightly. This is one of those "LaTeX things". For
% instance, "\textbf{A} \textbf{B}" will typeset as "A B" not "AB". To get
% "AB" then you have to do: "\textbf{A}\textbf{B}"
% \thanks is no different in this regard, so shield the last } of each \thanks
% that ends a line with a % and do not let a space in before the next \thanks.
% Spaces after \IEEEmembership other than the last one are OK (and needed) as
% you are supposed to have spaces between the names. For what it is worth,
% this is a minor point as most people would not even notice if the said evil
% space somehow managed to creep in.



% The paper headers
\markboth{Journal of \LaTeX\ Class Files,~Vol.~14, No.~8, October~2015}%
{Shell \MakeLowercase{\textit{et al.}}: 机器学习课程设计报告}
% The only time the second header will appear is for the odd numbered pages
% after the title page when using the twoside option.
% 
% *** Note that you probably will NOT want to include the author's ***
% *** name in the headers of peer review papers.                   ***
% You can use \ifCLASSOPTIONpeerreview for conditional compilation here if
% you desire.




% If you want to put a publisher's ID mark on the page you can do it like
% this:
%\IEEEpubid{0000--0000/00\$00.00~\copyright~2015 IEEE}
% Remember, if you use this you must call \IEEEpubidadjcol in the second
% column for its text to clear the IEEEpubid mark.



% use for special paper notices
%\IEEEspecialpapernotice{(Invited Paper)}




% make the title area
\maketitle

% As a general rule, do not put math, special symbols or citations
% in the abstract or keywords.
\begin{abstract}
随着人工智能的发展和交通安全需求的增加,基于深度学习的目标检测技术在行人头盔佩戴检测中发挥着重要用。在本研究中,针对行人检测场景采用YOLOv8l模型进行头盔检测,并通过稀疏化和剪枝技术对模型进行优化,在高精度的同时提升推理效率。利用稀疏化方法对YOLOv8l模型的权重进行优化,增强模型的稀疏性,减少计算冗余。通过剪枝技术有效去除冗余的网络参数和连接,大幅减少模型的计算量和参数规模,优化后的模型在嵌入式和边缘计算设备上运行效率显著提升。实验结果表明,经过稀疏化和剪枝后的YOLOv8l模型参数量减少了约30\%,推理速度提高了近40\%,并且在行人头盔检测任务中维持了高达98\%的准确率。我们还将优化后的模型与YOLOv3、YOLOv5、YOLOv7等多个版本进行了对比,验证了YOLOv8l在行人检测精度和速度方面的优势。
\end{abstract}

% Note that keywords are not normally used for peerreview papers.
\begin{IEEEkeywords}
交通管理,YOLOv8,头盔检测,深度学习,识别
\end{IEEEkeywords}

% For peer review papers, you can put extra information on the cover
% page as needed:
% \ifCLASSOPTIONpeerreview
% \begin{center} \bfseries EDICS Category: 3-BBND \end{center}
% \fi
%
% For peerreview papers, this IEEEtran command inserts a page break and
% creates the second title. It will be ignored for other modes.
\IEEEpeerreviewmaketitle

%1 引言
\section{引言}

%1.1课题背景及意义
\subsection{选题背景与意义}
Subsection text here.
%1.1.1课题背景
\subsubsection{课题背景}
subsubsection text here.
%1.1.1研究意义
\subsubsection{研究意义}
subsubsection text here.

%1.2研究问题分析
\subsection{研究问题分析}
Subsection text here.

%2 引言
\section{行人数据图像处理}
The conclusion goes here.

%2.1 实验数据集
\subsection{实验数据集}
Subsection text here.
%2.2 图像预处理
\subsection{图像预处理}
Subsection text here.
%2.3 yolo标准数据集
\subsection{yolo标准数据集}
Subsection text here.
%2.4 数据集标注
\subsection{数据集标注}
Subsection text here.

%3 基于YOLOv8的检测系统实现
\section{基于YOLOv8的检测系统实现}
The conclusion goes here.
%3.1 训练环境搭建
\subsection{训练环境搭建}
Subsection text here.

%3.2 模型训练
\subsection{模型训练}
Subsection text here.

%3.3 模型部署
\subsection{模型部署}
Subsection text here.

%4 基于YOLOv8的检测系统实现
\section{模型创新点}
The conclusion goes here.
%4.1 
\subsection{创新点1}
Subsection text here.
%4.2 
\subsection{创新点2}
Subsection text here.
%4.3 
\subsection{创新点3}
Subsection text here.

%5 实验结果
\section{实验结果}
The conclusion goes here.

%6 总结与展望
\section{总结与展望}
The conclusion goes here.

% if have a single appendix:
%\appendix[Proof of the Zonklar Equations]
% or
%\appendix  % for no appendix heading
% do not use \section anymore after \appendix, only \section*
% is possibly needed

% use appendices with more than one appendix
% then use \section to start each appendix
% you must declare a \section before using any
% \subsection or using \label (\appendices by itself
% starts a section numbered zero.)
%

% use section* for acknowledgment
\section*{致谢}


The authors would like to thank...


% Can use something like this to put references on a page
% by themselves when using endfloat and the captionsoff option.
\ifCLASSOPTIONcaptionsoff
  \newpage
\fi



% 参考文献
\begin{thebibliography}{1}

\bibitem{IEEEhowto:kopka}
111

\bibitem{IEEEhowto:kopka}
222

\end{thebibliography}


% that's all folks
\end{document}


