\documentclass[journal]{IEEEtran}

\usepackage{xeCJK} % CJK语言环境,使用XeLaTex进行编译
\usepackage{authblk} % 对应中文部分的作者机构特殊语法

% 将 authblk 包中的作者连词 and 替换为中文逗号
\renewcommand*{\Authsep}{,}
\renewcommand*{\Authand}{,}
\renewcommand*{\Authands}{,}

\setlength{\parindent}{2em} %2em代表首行缩进两个字符

%行间距
\usepackage{setspace} % 用于设置行间距

\onehalfspacing%设置1.5倍行距
%字号
\fontsize{22pt}{\baselineskip}{\selectfont}

%设置英文正文字体
\setmainfont{Times New Roman}
\usepackage{fontspec}

%图片
\usepackage{graphicx} %插入图片的宏包
\usepackage{float} %设置图片浮动位置的宏包
\usepackage{subfigure} %插入多图时用子图显示的宏包

\usepackage[justification=centering]{caption}

%段间距
\setlength{\parskip}{1pt}

% 设置图注编号为图1-1格式
\captionsetup[figure]{labelformat=simple, labelsep=quad, font={small, singlespacing}, justification=raggedright}

\captionsetup[table]{labelformat=simple, labelsep=quad, font={small, singlespacing}, justification=raggedright}

\renewcommand{\thefigure}{\arabic{section}-\arabic{figure}}

%引用
\usepackage{cite}

% correct bad hyphenation here
\hyphenation{op-tical net-works semi-conduc-tor}

\begin{document}

% paper title
\title{基于YOLOv8的头盔检测模型\\设计与实现}

\author{陈邵杰,郭昊,蔡明珠,刘政,黄俊毅,王晶昊,杨双菁,田睿朴,苗馨月% <-this % stops a space
\thanks{感谢电子科技大学机器学习课程闫老师与张老师的教学。}% <-this % stops a space
\thanks{2024年10月}}


% The paper headers
\markboth{Journal of \LaTeX\ Class Files,~Vol.~14, No.~8, October~2015}%
{Shell \MakeLowercase{\textit{et al.}}: 机器学习课程设计报告}


% make the title area
\maketitle

% As a general rule, do not put math, special symbols or citations
% in the abstract or keywords.
\begin{abstract}
随着人工智能的发展和交通安全需求的增加,基于深度学习的目标检测技术在行人头盔佩戴检测中发挥着重要用。在本研究中,针对行人检测场景采用YOLOv8l模型进行头盔检测,并通过稀疏化和剪枝技术对模型进行优化,在高精度的同时提升推理效率。利用稀疏化方法对YOLOv8l模型的权重进行优化,增强模型的稀疏性,减少计算冗余。通过剪枝技术有效去除冗余的网络参数和连接,大幅减少模型的计算量和参数规模,优化后的模型在嵌入式和边缘计算设备上运行效率显著提升。实验结果表明,经过稀疏化和剪枝后的YOLOv8l模型参数量减少了约30\%,推理速度提高了近40\%,并且在行人头盔检测任务中维持了高达98\%的准确率。我们还将优化后的模型与YOLOv3、YOLOv5、YOLOv7等多个版本进行了对比,验证了YOLOv8l在行人检测精度和速度方面的优势。
\end{abstract}

% Note that keywords are not normally used for peerreview papers.
\begin{IEEEkeywords}
交通管理,YOLOv8,头盔检测,深度学习,目标检测
\end{IEEEkeywords}

\IEEEpeerreviewmaketitle

%1 引言
\section{引言}

%1.1课题背景及意义
\subsection{选题背景与意义}
Subsection text here.
%1.1.1课题背景
\subsubsection{课题背景}
subsubsection text here.
%1.1.1研究意义
\subsubsection{研究意义}
subsubsection text here.

%1.2研究问题分析
\subsection{研究问题分析}
Subsection text here.

%2 引言
\section{行人数据图像处理}
The conclusion goes here.

%2.1 实验数据集
\subsection{实验数据集}
Subsection text here.A'
%2.2 图像预处理
\subsection{图像预处理}
Subsection text here.
%2.3 yolo标准数据集
\subsection{yolo标准数据集}
Subsection text here.
%2.4 数据集标注
\subsection{数据集标注}
Subsection text here.

%3 基于YOLOv8的检测系统实现
\section{基于YOLOv8的检测系统实现}
YOLOv8是一种精准高效的目标检测解决方案,系统采用YOLOv8作为核心检测算法,识别电动车及其驾驶员,并判断是否存在佩戴头盔行为。训练与测试使用的均为540*355的YOLO标准数据集,图3-1为整个实验流程图:
\begin{figure}[htbp] 

   \centering
   \includegraphics[width=8cm]{figures/流程图.png}
   \caption{实验流程图} 
   \label{fig:} 
  
\end{figure} 

%3.1 训练环境搭建
\subsection{训练环境搭建}
对于大量图片数据集与复杂模型,通过GPU进行模型训练大加速,通过AutoDL平台租借云服务器实现YOLOv8模型的搭建,训练,测试与部署工作。\par
本次实验使用的硬件环境为:CPU 16 vCPU Intel(R) Xeon(R) Platinum 8481C,GPU RTX 4090D,显存大小24G,硬盘为80G。\par
软件环境:操作系统ubutnu22.04,编程语言python版本3.8.5,pytorch版本1.10.0,Cuda版本11.8。\par

%3.2 模型训练
\subsection{模型训练}
本次实验采用YOLOv8模型进行目标检测训练,整个训练过程依托于高性能服务器硬件及成熟的软件环境,确保了模型的高效收敛和稳定性能。\par
训练流程从数据准备开始,输入图像维度为像素540*355,首先对数据集进行了精细的标注、清洗和划分,形成训练集、验证集和测试集,为模型的训练和评估提供了坚实的数据基础。并且通过高性能硬件配置使得数据处理与模型计算可以高效并行执行,显著提高了训练效率。\par
整个训练过程共进行了100个epoch,训练阶段首先加载数据集,并进行数据增强操作,提高模型的泛化能力。接着初始化YOLOv8模型,设置包括学习率和批次大小等超参数,以确保训练初期能够有效学习特征。每个epoch包含前向传播、损失计算、反向传播和权重更新等步骤。\par
在服务器上,GPU的强大算力极大加速了这些运算,在卷积层的特征提取和参数更新过程中尤为显著。每个epoch结束后,模型会在验证集上进行评估,以监控性能并避免过拟合的发生。训练日志详细记录了每个epoch的损失、精度和召回率等关键指标,并且实时生成了损失曲线和精度曲线,以便于监控训练过程中的动态表现和分析模型的改进方向。\par
在进行模型参数配置时,为了确保对比实验的一致性和准确性,我们需要核心参数固定不变,以排除对训练效果产生的潜在影响。训练epoch为100轮,使用常见的随机梯度下降优化器,初始学习率和周期学习率均设置为0.01,设置batch数量为64。经训练,YOLOv8模型最终达到了预期的检测效果,能够精准地识别目标物体,表现出良好的鲁棒性和检测性能。通过数据预处理、模型初始化、迭代训练和模型评估的完整闭环与高性能服务器硬件的支持,使得模型在合理时间内完成了高效训练。\par

%3.3 模型部署
\subsection{模型部署}
在本项目中,我们将YOLOv8目标检测模型部署到基于PySide6开发的图形化界面应用中。模型部署的核心步骤包括开发UI界面,将经过训练的YOLOv8模型进行格式转换等。\par
在应用程序中,用户能够通过直观的界面上传图像或视频流,并实时查看检测结果。这一过程中,我们充分利用了PySide6的强大功能,创建了用户友好的操作面板,使用户体验更加流畅。此外,部署过程中还进行了全面的测试,以确保在不同输入场景下的稳定性和可靠性。通过此次部署,YOLOv8模型得以在实际应用中发挥重要作用,推动了目标检测技术的实用化进程。\par
\begin{figure}[htbp] 

   \centering
   \includegraphics[width=8cm]{figures/部署页面.png}
   \caption{部署界面} 
   \label{fig:} 
  
\end{figure} 

%4 调整与改进
\section{调整与改进}


%5 实验结果
\section{实验结果}
The conclusion goes here.

%6 总结与展望
\section{总结与展望}
The conclusion goes here.

% if have a single appendix:
%\appendix[Proof of the Zonklar Equations]
% or
%\appendix  % for no appendix heading
% do not use \section anymore after \appendix, only \section*
% is possibly needed

% use appendices with more than one appendix
% then use \section to start each appendix
% you must declare a \section before using any
% \subsection or using \label (\appendices by itself
% starts a section numbered zero.)
%

% use section* for acknowledgment
\section*{致谢}


The authors would like to thank...


% Can use something like this to put references on a page
% by themselves when using endfloat and the captionsoff option.
\ifCLASSOPTIONcaptionsoff
  \newpage
\fi



% 参考文献
\begin{thebibliography}{1}

\bibitem{1}
Woo S, Park J, Lee J Y, et al. CBAM: Convolutional block attention module[C]//Proceedings of the European conference on computer vision (ECCV). 2018: 3-19.

\bibitem{2}
Vaswani A. Attention is all you need[J]. Advances in Neural Information Processing Systems, 2017.

\end{thebibliography}


% that's all folks
\end{document}


